% !TeX root = ../main.tex
% Add the above to each chapter to make compiling the PDF easier in some editors.
\chapter{Prelimiaries}\label{chapter:prelimiaries}

\theoremstyle{definition}
\newtheorem{definition}{Definition}[section]

\section{Mathematical Prelimiaries}

\subsection{General Notation}
\begin{itemize}
    \item p stands for a prime number if not specified otherwise.
    \item groups are written in a multiplicative way with the common operator $\cdot$ and the abbreviation $ab$ for $a \cdot b$
    \item $\mathbb{Z}_p$ denotes a finite field of order p, which is isomorph to the integers modulo p \parencite{algebra}, hence we use the common integer operators for addition and multiplication. 
\end{itemize}


\subsection{cyclic group}
A finite group $\mathbb{G}_p$ of order p with a generator element is a cyclic group \parencite{algebra}.

We use \textbf{g} to refer to a fixed generator of a cyclic group. Since \textbf{g} is a generator
$\{1,\textbf{g},\textbf{g}^2,...,\textbf{g}^{p-1}\}$ is isomorph to $\mathbb{G}_p$ and the power operation is closed on $\mathbb{G}_p$ \parencite{algebra}.


\subsection{pairings}
A pairing is a function: $\mathbb{G}_p \times \mathbb{G}_p \rightarrow \mathbb{G}_T$, where $\mathbb{G}_p$ and $\mathbb{G}_T$ are two groups of order p \parencite{KZG}. From now on $e$ will always denote a pairing function.
Pairings require two properties \parencite{KZG}:
\begin{itemize}
    \item \textbf{Billinearity:} $\forall g,h \in \mathbb{G}_p.\ \forall a,b \in \mathbb{Z}_p.\ e(g^a,h^b) = e(g,h) ^{ab}$
    \item \textbf{Non-degeneracy:} $\neg (\forall g,h  \in \mathbb{G}_p.\  e(g,h) = 1)$
\end{itemize}
  
\section{Cryptography Prelimiaries}

Before we introduce the cryptographic preliminaries, we cover some general notation that is used in this paper in the context of cryptography.
\begin{itemize}
    \item To express that an element is uniformly sampled from a set, we use the abbreviation '$\in_\mathcal{R}$'. 
    \item $\epsilon$ is a function that is considered negligible in the security parameter $\kappa$, 
    where negligibility means that for all $c>0$ there exists a $k_0$ such that $\epsilon(k) < 1/k^c$ for all $k > k_0$
\end{itemize}

Note that further topic-related notation (e.g. for games) is to be found in the according topic's section.

\subsection{Game-based Proofs}
Games are a method to define security for cryptographic protocols, they are composed of probabilistic functions and played against a probabilistic Adversary \parencite{shoup_games}. Bellare, Rogaway and Shoup state that game-based proofs are a particularly rigor and thus secure proving approach, referring to game-based proofs as a sequence of game-hops that bound the probability of one game to another \parencite{gamesB&R,shoup_games}. The two types of game hops we will use in our proofs are: 
\begin{itemize}
    \item \textbf{game hop as a bridging step} \\
    A bridging step is changing the function definitions, such that the game's probability does not change \parencite{shoup_games}. 
    \item \textbf{game hop based on a failure event} \\
    In a game hop based on a failure event, two games are equal except if a specific failure event occurs \parencite{shoup_games}. The failure event should have a negligible probability for the game-based proof to hold. 
\end{itemize}
We write games as a sequence of functions where '$\leftarrow$' followed by a set means uniform sampling from that set, '$\leftarrow$' followed by a probability mass function means sampling from that function space, and '=' is an assignment of a deterministic value. Moreover, we write ':' followed by a condition to assure that the condition has to hold at this point. 
To give an example, think of the following game as "sampling a uniformly random $a$ from $\mathbb{Z}_p$, get the probabilistic result from $\mathcal{A}$ as b, computing c as F applied to a and b, and assert that P holds for c":
\begin{equation*}
    \begin{split}
        a & \leftarrow \mathbb{Z}_p,\\
        b & \leftarrow \mathcal{A},\\
        c & = \text{F }a \ b,\\
        & : \text{P } c 
    \end{split}
\end{equation*}



\subsection{Hardness Assumptions}
Hardness assumptions are problems that are generally assumed to be hard. Security games for cryptographic protocols are bound to hardness assumptions via game hops to obtain game-based security proofs \parencite{boneh_shoup}.
We will need three specific hardness assumptions for our proofs, all of which are defined in \parencite{KZG}:

\begin{definition}[Discrete Logarithm (DL) Assumption]
    For $a \in_\mathcal{R} \mathbb{Z}_p$, holds for every Adversary $\mathcal{A}$: Pr$[a=\mathcal{A}(\textbf{g}^a)] = \epsilon$ \parencite{KZG}.

    Formally We define the DL game as:
    \begin{equation*}
        \begin{split}
            a & \leftarrow \mathbb{Z}_p,\\
            a' & \leftarrow \mathcal{A}\ \textbf{g}^a,\\
            & : a = a'
        \end{split}
    \end{equation*}
\end{definition}


\subsubsection{t-strong Diffie-Hellman (t-SDH) Assumption}
Let $t$ be fixed. For $\alpha \in_\mathcal{R} \mathbb{Z}_p$, holds for every Adversary $\mathcal{A}$: Pr$\Big[
    (c,\textbf{g}^{\frac{1}{\alpha+c}})=\mathcal{A}\ [\textbf{g}, \textbf{g}^\alpha, \textbf{g}^{(\alpha^2)}, \dots, \textbf{g}^{(\alpha^t)}]
    \Big] 
= \epsilon$ for all c $\in \mathbb{Z}_p\backslash\{\alpha\}$ \parencite{KZG}.

Formally We define the t-SDH game as:
\begin{equation*}
    \begin{split}
        \alpha & \leftarrow \mathbb{Z}_p,\\
        (c, g') & \leftarrow  \mathcal{A}\ [\textbf{g}, \textbf{g}^\alpha, \textbf{g}^{(\alpha^2)}, \dots, \textbf{g}^{(\alpha^t)}]\\
        & : \textbf{g}^{\frac{1}{\alpha+c}} = g'
    \end{split}
\end{equation*}

\subsubsection{t-Bilinear Strong Diffie-Hellman (t-BSDH) Assumption}
This definition is analogous to the previous one, except that the result is passed through a pairing function. Nevertheless, we define the property formally for completeness. 

Let $t$ be fixed. For $\alpha \in_\mathcal{R} \mathbb{Z}_p$, holds for every Adversary $\mathcal{A}$: Pr$\Big[
    (c,\text{e}(\textbf{g}, \textbf{g})^{\frac{1}{\alpha+c}})=\mathcal{A}\ [\textbf{g}, \textbf{g}^\alpha, \textbf{g}^{(\alpha^2)}, \dots, \textbf{g}^{(\alpha^t)}]
    \Big] 
= \epsilon$ for all c $\in \mathbb{Z}_p\backslash\{\alpha\}$ \parencite{KZG}.

Formally We define the t-BSDH game as:
\begin{equation*}
    \begin{split}
        \alpha & \leftarrow \mathbb{Z}_p,\\
        (c, g') & \leftarrow  \mathcal{A}\ [\textbf{g}, \textbf{g}^\alpha, \textbf{g}^{(\alpha^2)}, \dots, \textbf{g}^{(\alpha^t)}]\\
        & : \text{e}(\textbf{g}, \textbf{g})^{\frac{1}{\alpha+c}} = g'
    \end{split}
\end{equation*}

\subsection{Commitment Schemes}
~\parencite{thalerbook}

\section{Isabelle Prelimiaries}

\subsection{Isabelle based Notation}

\subsection{CryptHOL}